% Options for packages loaded elsewhere
\PassOptionsToPackage{unicode}{hyperref}
\PassOptionsToPackage{hyphens}{url}
\PassOptionsToPackage{dvipsnames,svgnames,x11names}{xcolor}
%
\documentclass[
  number]{elsarticle}

\usepackage{amsmath,amssymb}
\usepackage{iftex}
\ifPDFTeX
  \usepackage[T1]{fontenc}
  \usepackage[utf8]{inputenc}
  \usepackage{textcomp} % provide euro and other symbols
\else % if luatex or xetex
  \usepackage{unicode-math}
  \defaultfontfeatures{Scale=MatchLowercase}
  \defaultfontfeatures[\rmfamily]{Ligatures=TeX,Scale=1}
\fi
\usepackage{lmodern}
\ifPDFTeX\else  
    % xetex/luatex font selection
\fi
% Use upquote if available, for straight quotes in verbatim environments
\IfFileExists{upquote.sty}{\usepackage{upquote}}{}
\IfFileExists{microtype.sty}{% use microtype if available
  \usepackage[]{microtype}
  \UseMicrotypeSet[protrusion]{basicmath} % disable protrusion for tt fonts
}{}
\makeatletter
\@ifundefined{KOMAClassName}{% if non-KOMA class
  \IfFileExists{parskip.sty}{%
    \usepackage{parskip}
  }{% else
    \setlength{\parindent}{0pt}
    \setlength{\parskip}{6pt plus 2pt minus 1pt}}
}{% if KOMA class
  \KOMAoptions{parskip=half}}
\makeatother
\usepackage{xcolor}
\setlength{\emergencystretch}{3em} % prevent overfull lines
\setcounter{secnumdepth}{5}
% Make \paragraph and \subparagraph free-standing
\makeatletter
\ifx\paragraph\undefined\else
  \let\oldparagraph\paragraph
  \renewcommand{\paragraph}{
    \@ifstar
      \xxxParagraphStar
      \xxxParagraphNoStar
  }
  \newcommand{\xxxParagraphStar}[1]{\oldparagraph*{#1}\mbox{}}
  \newcommand{\xxxParagraphNoStar}[1]{\oldparagraph{#1}\mbox{}}
\fi
\ifx\subparagraph\undefined\else
  \let\oldsubparagraph\subparagraph
  \renewcommand{\subparagraph}{
    \@ifstar
      \xxxSubParagraphStar
      \xxxSubParagraphNoStar
  }
  \newcommand{\xxxSubParagraphStar}[1]{\oldsubparagraph*{#1}\mbox{}}
  \newcommand{\xxxSubParagraphNoStar}[1]{\oldsubparagraph{#1}\mbox{}}
\fi
\makeatother


\providecommand{\tightlist}{%
  \setlength{\itemsep}{0pt}\setlength{\parskip}{0pt}}\usepackage{longtable,booktabs,array}
\usepackage{calc} % for calculating minipage widths
% Correct order of tables after \paragraph or \subparagraph
\usepackage{etoolbox}
\makeatletter
\patchcmd\longtable{\par}{\if@noskipsec\mbox{}\fi\par}{}{}
\makeatother
% Allow footnotes in longtable head/foot
\IfFileExists{footnotehyper.sty}{\usepackage{footnotehyper}}{\usepackage{footnote}}
\makesavenoteenv{longtable}
\usepackage{graphicx}
\makeatletter
\newsavebox\pandoc@box
\newcommand*\pandocbounded[1]{% scales image to fit in text height/width
  \sbox\pandoc@box{#1}%
  \Gscale@div\@tempa{\textheight}{\dimexpr\ht\pandoc@box+\dp\pandoc@box\relax}%
  \Gscale@div\@tempb{\linewidth}{\wd\pandoc@box}%
  \ifdim\@tempb\p@<\@tempa\p@\let\@tempa\@tempb\fi% select the smaller of both
  \ifdim\@tempa\p@<\p@\scalebox{\@tempa}{\usebox\pandoc@box}%
  \else\usebox{\pandoc@box}%
  \fi%
}
% Set default figure placement to htbp
\def\fps@figure{htbp}
\makeatother

\usepackage{booktabs}
\usepackage{caption}
\usepackage{longtable}
\usepackage{colortbl}
\usepackage{array}
\usepackage{anyfontsize}
\usepackage{multirow}
\makeatletter
\@ifpackageloaded{float}{}{\usepackage{float}}
\floatstyle{plain}
\@ifundefined{c@chapter}{\newfloat{suppfig}{h}{losuppfig}}{\newfloat{suppfig}{h}{losuppfig}[chapter]}
\floatname{suppfig}{Figure S}
\newcommand*\quartosuppfigref[1]{Figure \hyperref[#1]{S\ref{#1}}}
\@ifpackageloaded{caption}{}{\usepackage{caption}}
\DeclareCaptionLabelFormat{quartosuppfigreflabelformat}{#1#2}
\captionsetup[suppfig]{labelformat=quartosuppfigreflabelformat}
\newcommand*\listofsuppfigs{\listof{suppfig}{List of Supplementary Figures}}
\makeatother
\makeatletter
\@ifpackageloaded{float}{}{\usepackage{float}}
\floatstyle{plain}
\@ifundefined{c@chapter}{\newfloat{supptab}{h}{losupptab}}{\newfloat{supptab}{h}{losupptab}[chapter]}
\floatname{supptab}{Table S}
\newcommand*\quartosupptabref[1]{Table \hyperref[#1]{S\ref{#1}}}
\@ifpackageloaded{caption}{}{\usepackage{caption}}
\DeclareCaptionLabelFormat{quartosupptabreflabelformat}{#1#2}
\captionsetup[supptab]{labelformat=quartosupptabreflabelformat}
\newcommand*\listofsupptabs{\listof{supptab}{List of Supplementary Tables}}
\makeatother
\makeatletter
\@ifpackageloaded{caption}{}{\usepackage{caption}}
\AtBeginDocument{%
\ifdefined\contentsname
  \renewcommand*\contentsname{Table of contents}
\else
  \newcommand\contentsname{Table of contents}
\fi
\ifdefined\listfigurename
  \renewcommand*\listfigurename{List of Figures}
\else
  \newcommand\listfigurename{List of Figures}
\fi
\ifdefined\listtablename
  \renewcommand*\listtablename{List of Tables}
\else
  \newcommand\listtablename{List of Tables}
\fi
\ifdefined\figurename
  \renewcommand*\figurename{Figure}
\else
  \newcommand\figurename{Figure}
\fi
\ifdefined\tablename
  \renewcommand*\tablename{Table}
\else
  \newcommand\tablename{Table}
\fi
}
\@ifpackageloaded{float}{}{\usepackage{float}}
\floatstyle{ruled}
\@ifundefined{c@chapter}{\newfloat{codelisting}{h}{lop}}{\newfloat{codelisting}{h}{lop}[chapter]}
\floatname{codelisting}{Listing}
\newcommand*\listoflistings{\listof{codelisting}{List of Listings}}
\makeatother
\makeatletter
\makeatother
\makeatletter
\@ifpackageloaded{caption}{}{\usepackage{caption}}
\@ifpackageloaded{subcaption}{}{\usepackage{subcaption}}
\makeatother

\usepackage[]{natbib}
\bibliographystyle{elsarticle-num}
\usepackage{bookmark}

\IfFileExists{xurl.sty}{\usepackage{xurl}}{} % add URL line breaks if available
\urlstyle{same} % disable monospaced font for URLs
\hypersetup{
  pdftitle={Investigating the Spatial Variability in Soil Geochemical and Colour Properties Across Two Contrasting Land Uses in South-Central Manitoba},
  pdfauthor={Maria Luna Miño; Alexander J Koiter; Taras E Lychuk; Arnie Waddel; Alan Moulin},
  pdfkeywords={Soil geochemistry, Soil colour, Spatial analysis, Terrain
attributes},
  colorlinks=true,
  linkcolor={blue},
  filecolor={Maroon},
  citecolor={Blue},
  urlcolor={Blue},
  pdfcreator={LaTeX via pandoc}}


\setlength{\parindent}{6pt}
\begin{document}

\begin{frontmatter}
\title{Investigating the Spatial Variability in Soil Geochemical and
Colour Properties Across Two Contrasting Land Uses in South-Central
Manitoba}
\author[1]{Maria Luna Miño%
%
}
 \ead{LUNAMIMA56@brandonu.ca} 
\author[2]{Alexander J Koiter%
\corref{cor1}%
}
 \ead{koitera@brandonu.ca} 
\author[3]{Taras E Lychuk%
%
}
 \ead{taras.lychuk@AGR.GC.CA} 
\author[3]{Arnie Waddel%
%
}
 \ead{arnie.waddell@AGR.GC.CA} 
\author[3]{Alan Moulin%
%
}
 \ead{apmaafc7788@gmail.com} 

\affiliation[1]{organization={Brandon University, Masters in
Environmental and Life Sciences},addressline={270 18th
St},city={Brandon},postcode={R7A 6A9},postcodesep={}}
\affiliation[2]{organization={Brandon University, Department of
Geography and Environment},addressline={270 18th
St},city={Brandon},postcode={R7A 6A9},postcodesep={}}
\affiliation[3]{organization={Agriculture and Agri-Food Canada, Brandon
Research and Development Centre},addressline={2701 Grand Valley
Road},city={Brandon},postcode={R7A 5Y3},postcodesep={}}

\cortext[cor1]{Corresponding author}





        
\begin{abstract}
Quantification and accurate assessment of the spatial variability and
distribution of soil physical and biogeochemical properties are vital
components of agri-environmental research and modeling, including
sediment source fingerprinting. Understanding the distribution of soil
properties is crucial in the development of appropriate, reliable, and
efficient sampling campaigns. This study was aimed at investigating the
spatial variability in soil geochemical and colour (i.e., spectral
reflectance) soil properties (\textless63um) across two contrasting land
uses. The main objectives of this study are to: 1) quantify the spatial
variability of geochemical and colour properties at a field-scale
(\textasciitilde{} 40 ha) across agricultural and forested sites; 2)
assess the spatial variability and distribution of soil properties and
its relation to seven terrain attributes (e.g., catchment area,
elevation). A combination of univariate analysis and geostatistical
methods were applied to characterize the soil geochemistry and colour
properties. This information was used to both quantify and assess the
variability in soil properties. The variability and spatial
autocorrelation were generally both site and soil property specific. For
a selection of soil properties exhibiting some spatial autocorrelation,
random forest regression was used to identify the relative importance of
terrain attributes on observed patterns of soil geochemical and colour
properties. Elevation was found to explain the greatest amount of the
variation in soil properties followed by the SAGA wetness index and
relative slope position. These findings can be used to help create
efficient soil sampling designs by providing information that can inform
sampling locations and number of samples collected in order to meet
research needs and objectives.
\end{abstract}





\begin{keyword}
    Soil geochemistry \sep Soil colour \sep Spatial analysis \sep 
    Terrain attributes
\end{keyword}
\end{frontmatter}
    

\subsection{Introduction}\label{introduction}

Variation in soil biological, chemical, and physical properties occurs
across the landscape and in response to both regional and local (i.e.,
field-scale) variations in the five soil forming factors: parent
material, relief or topography, biota, climate, and time. Superimposed
on this is the influence of changes in land use and current and historic
management practices which can further modify soil properties.
Quantifying and understanding the patterns and drivers for this
variation is an important component of many agri-environmental studies.
For example, to meet the desired level of precision for agronomic and
environmental nutrient management plans the spatial variability in soil
nutrients will influence the soil sampling design in terms of number and
locations of soil samples \citep{starr1995, kariuki2009}.

Sediment source fingerprinting is a watershed-scale technique that is
used to identify and quantify the relative proportions of sediment
derived from unique sources. This technique uses natural occurring
biogeochemical properties as fingerprints (i.e., tracers) to
discriminate between potential sources of sediment and are linked to
downstream sediment using mixing models. From a sediment fingerprinting
perspective, investigating the spatial variability of soil properties at
a watershed-scale can be advantageous to identify, classify, and
distinguish between potential sources of sediment \citep{pulley2017}.
However, investigating spatial variability at smaller scales is less
common \citep[e.g.,][]{du2017, pulley2018, collins2019, lunamiño2024}
and remains a research priority \citep{collins2020}.

There are three main, interconnected, ways that spatial variability in
fingerprint properties are an important aspect of sediment
fingerprinting. First is to adequately quantify the fingerprint
properties such that it is representative of that source. For some
fingerprints the variability is not random but rather varies in a more
systematic way. For example, the pattern of fallout radionuclides will
reflect the patterns of soil erosion and deposition
\citep{wilkinson2015}. Designing and implementing source sampling plans
need to take this into consideration as the sampling designed used has
been shown to influence the characterization of wide range of commonly
used fingerprints \citep{lunamiño2024}.

Secondly, the issue of spatial variability of fingerprint properties is
further complicated by overlying spatial variability in the rates of
erosion and sediment delivery. Incorporation of both types of
variability into the mixing model will provide a more reliable estimate
of the proportion of sediment derived from each source. Many mixing
models have well defined inputs (sources) and outputs (sediment) that
are characterized by their mean and standard deviation and the spatial
distribution or pattern of fingerprints are not considered. This is not
ideal as the values of samples that are collected closer, and more
hydrologically connected, to the stream network may in fact present a
better representation of that source despite potentially deviating from
the mean value. This issue can be addressed by strategic sampling where
the more likely to erode areas are targeted for sampling. However, a
considerable amount of information and insight is lost through that
approach. There has been some progress using information on erosion
rates to calculate a erosion rate-weighted mean
\citep{wilkinson2015, du2017} and using spatially interpolated maps of
fingerprint values to provide a finer resolution of the fingerprint
variability within each source \citep{haddadchi2019}.

Lastly, understanding the geomorphic, hydrologic, and biochemical
processes that have led to the observed patterns in spatial variability
helps in the selection of robust and reliable fingerprints and/or guide
the sampling design for source characterization. In selecting
fingerprints that provide good discrimination between sources many
studies typically used a statistical-based approach \citep{collins1997}.
However, there are concerns that this approach may result in the
inclusion of false positives (i.e., type I error) or non-conservative
fingerprints \citep{koiter2013}. Consequently, there has been a call for
the inclusion of a process-based (e.g., weathering, erosion) or
geologic/lithologic-based explanation of the fingerprints selected to
address these concerns \citep{collins2020}. Furthermore, there is also a
lack of standardization in how sediment source areas are sampled (e.g.,
judgement, random, transect, grid, stratified) and it can be difficult
to have an efficient sampling design without prior knowledge of why and
how soil properties vary across the landscape \citep{lunamiño2024}.
Prior knowledge of the spatial variability of soil fingerprint
properties would be beneficial; however, in practice this can be
difficult, particularly with geochemical properties as routine lab
analysis often return information on more than 50 elements. The spatial
patterns of some soil properties are well studied because of their
agronomic importance or ability to infer other important soil properties
and processes and can include fallout radionuclides {[}e.g.,
\textsuperscript{137}Cs, \textsuperscript{7}Be; \citep{ritchie1970}{]},
plant nutrients {[}e.g., N, P; \citep{vasu2017}{]}, soil colour {[}e.g.,
hue, value; \citep{viscarrarossel2006}{]}, major non-acid forming
cations {[}e.g., Ca, Na; \citep{sun2021}{]}. In contrast, the processes
that control the distribution of other soil properties, such as rare
earth elements and trace metals, are less well studied or tend to be
site-specific, making it difficult to draw generalizations.

Terrain attributes such as elevation, slope curvature, slope position,
and soil wetness indices have been shown to be useful information in the
understanding and modelling of a range of soil properties including soil
moisture \citep{beaudette2013}, texture \citep{kokulan2018}, colour
\citep{brown2004}, organic matter \citep{zhang2012}, conductivity
\citep{umali2012}, and geochemistry \citep{lima2023}. Similar techniques
may provide additional insight into the pedologic and geomorphic
processes that drive the observed patterns of fingerprint properties
within a given source. Since digital elevation models (DEMs) are more
publicly available and can in some case be generated using drone
imagery, while soil property data are often limited, terrain attributes
derived from DEMs can be used to guide sampling design.

This study builds on the previous work of Luna Miño \citep{lunamiño2024}
where the impact of three different sampling designs on the
characterization of source materials, within the framework of the
sediment fingerprinting approach, was assessed. This study expands that
study by using the data from grid sampling approach to assess the
spatial autocorrelation, create soil property (i.e., fingerprint) maps,
and identify important terrain attributes driving the observed patterns.
The objectives of this study were (1) to investigate the spatial
variability of a range of soil colour and geochemical properties in an
agricultural and forested site; and (2) to assess the relative
importance and correlation of terrain attributes with the spatial
distribution of these soil properties. Together, these objectives
address how terrain attributes may be used to understand spatial
distributions of soil properties and help guide sampling design.

\subsection{Methods}\label{methods}

\subsubsection{Site description}\label{site-description}

Two sites of contrasting land uses located in the Wilson Creek Watershed
(WCW), near McCreary, Manitoba, Canada were selected to investigate the
spatial variability in fingerprint properties. The headwaters of the WCW
are located on top of the Manitoba Escarpment within the boundary of
Riding Mountain National Park. There is a \textasciitilde300m drop in
elevation crosses the escarpment where the streams become deeply
incised. At the base of the escarpment is a large alluvial fan situated
in the lacustrine deposits of glacial lake Aggasiz where the main stem
has a meandering form. However, beyond the national park boundary the
stream flows straight through an engineered drain until it reaches the
Turtle river (Figure~\ref{fig-location_map}). Both sites are both
hydrologicaly connected to the mainstem of the Wilson Creek

The first site was a mixedwood forest including white and black spruce
(Picea glauca, Picea mariana), balsam fir (Abies balsamea), larch (Larix
laricina) and young stands of deciduous trees including trembling aspen
(Populus tremuloides). The forested site is located within the
boundaries of the national park where there is little disturbance beyond
recreational hiking trails. The soils within the park are not well
mapped but likely are part of the Grey Wooded soil association (Luvisol)
consisting of fine sandy loam to clay loam soils developed on boulder
till of mostly shale with some limestone, and granitic rocks
\citep{ehrlich1958}. The second site is under agricultural production
and includes rotations of grain crops and forage. The site is mapped to
the Edwards Soil Series (Cumulic Regosol) consisting of silty clay loam
to silty clay soil developed on recent alluvial deposits
\citep{ehrlich1958}.

The Köppen-Geiger climate classification of the WCW is cold, without dry
season, and with warm summer (Dfb) \citep{beck2018}. The average annual
precipitation is \textasciitilde539 mm, with approximately 27\% falling
as snow with a mean annual temperature is 3.0°C
\citep{environmentandclimatechangecanada2024}. The hydrology of the
watershed is snowmelt dominated with \textasciitilde{} 80\% of the
cumulative runoff occurring during the spring season (May and June)
\citep{mackay1970}.

\begin{figure}[H]

\centering{

\pandocbounded{\includegraphics[keepaspectratio]{index_files/figure-latex/notebooks-location_map-fig-location_map-output-2.png}}

}

\caption{\label{fig-location_map}a) Map showing the location of the
study sites within Canada. Location of the two study sites and nearby
town of McCreary, and regional b) land use, and c) topography.}

\end{figure}%

\textsubscript{Source:
\href{https://alex-koiter.github.io/spatial-variability-soil-manuscript/notebooks/location_map.qmd.html\#cell-fig-location_map}{Research
Site Locations}}

\subsubsection{Soil sampling and
analysis}\label{soil-sampling-and-analysis}

This study uses samples and data collected as part of the grid sampling
design outlined in Luna Miño \citep{lunamiño2024}. Briefly, at each site
49 samples were collected using a soil auger on a 7x7 grid at a 100m
spacing. Within the forested surface soil samples were collected below
the LFH layer to a depth of 5cm, and the agricultural site was sampled
to a depth of 15cm to account for the regular mixing of the soil due to
tillage and other field operations.

Samples were dried, homogenized with a mortar and pestle, and sieved
through a 63 𝜇m sieve to remove the sand fraction. The sand fraction was
removed in an effort to reduce the differences in grain size and organic
matter content between the two sites \citep{laceby2017}. Samples were
analyzed for a broad suite geochemical element using inductively coupled
plasma mass spectrometry (ICP-MS) following a microwave-assisted
digestion with aqua-regia (ALS Mineral Division, North Vancouver, BC,
Canada). Spectral measurements were made with a spectroradiometer (ASD
FieldSpecPro Malvern Panalytical Inc Westborough MA 01581, United
States). Spectral reflectance measurements were taken in 1 nm increments
over the 0.4-2.5 μm wavelength range. Both samples and Spectralon
standard (white reference) were illuminated with a white light source
using a halogen-based lamp (12 VDC, 20 Watt). Following the method
outlined in Boudreault et al. \citep{boudreault2018}, fifteen colour
coefficients (R, G, B, x, y, Y, X, Z, L, a*, b*, u*, v*, c*, h*) were
calculated for each sample \citep{koiter2021}. Based on the results of
Luna Miño \citep{lunamiño2024}, a composite fingerprint consisting of 10
geochemical elements (Ca, Co, Cs, Fe, Li, La, Nb, Ni, Rb, and Sr) and
five colour coefficients (a*, b*, c*, h*, and x) were identifying as
providing a strong discrimination between the agricultural and forested
surface soils. These fifteen soil properties are the focus of the
detailed spatial analysis detailed in this study.

\subsubsection{Geospatial and terrain
analysis}\label{geospatial-and-terrain-analysis}

All geostatistics were performed with ArcGIS Pro \citep[v
3.3.0][]{esri2024}. Semivariograms were used to quantify spatial
correlation for each of the 15 soil properties. The optimization tool,
based on minimizing the mean square error, was used to parameterize the
semivariogram model. Kriging was used to interpolate and generate maps
of each soil property. The exploratory interpolation tool
(Geostatistical Analyst extension) was used to select the kriging type
with the highest ranked prediction accuracy.

A Digital Elevation Model (DEM) for the forested site was acquired from
publicly available data \citep{naturalresourcescanada2024}. A DEM for
the agricultural site was generated by photogrammetry using UAV imagery,
including the use of ground control and check points, with Agisoft
Metashape Professional \citep[v1.8.2][]{agisoft2021}. Ordinary kriging
was used to calculate a 1 m gridded digital elevation model for each
site. Geographic information software \citep[SAGA v2.1.4][]{conrad2015}
was used to calculate six additional terrain attributes and included
plan and profile curvatures, saga wetness index, catchment area,
relative slope position, and vertical channel network distance
(\quartosupptabref{supptab-terrain}).

\subsubsection{Data analysis}\label{data-analysis}

All subsequent statistical analysis was undertaken using R statistical
Software v4.4.0 \citep{rcoreteam2024} through RStudio Integrated
Development Environment v2024.04.2 \citep{rstudio2024}. Plots and maps
were created using the R package \texttt{ggplot2} v 3.5.2
\citep{wickham2016}. Skewness was categorized as values between -0.5 and
0.5 considered approximately symmetric, -1.0 to -0.5 or 0.5 to 1 as
moderately skewed, and \textless{} -1.0 or \textgreater{} 1.0 as highly
skewed. Coefficient of variation (CV) thresholds were categorized as low
(\textless15\%), moderate (15--35\%), high (35--75\%), and very high
(\textgreater75\%). Interpolated soil property and terrain attribute
data were resampled to a 10 m resolution prior to analysis
\citep[\texttt{terra} v1.8.29][]{hijmans2024}. Random Forest Regression
\citep[\texttt{randomForest} v4.7.1.2][]{liaw2002} was used to assess
the relative importance of the terrain attributes on the spatial
distribution of soil properties. The dataset was randomly split into
training, validation, and testing datasets. Multicollinearity among the
terrain attributed was assessed using the Variance Inflation Factor with
a threshold of eight and correlated terrain attributes were removed
\citep[\texttt{usdm} v2.1.7][]{Naimi2014}. The number of variables
randomly sampled as candidates at each split within the random forest
model was tuned using the training and validation data sets
\citep[\texttt{caret} v7.0.1][]{kuhn2008}. The number of trees to grow
was set to 500 and model performance was assessed using the Mean Square
Error (MSE) and percent variance explained for both the training (Out of
Bag Error) and the validation data sets. To test the model, actual and
predicted values were plotted and the
R\textsuperscript{\textsubscript{2}} and MSE were calculated using the
testing data set. Because analyzing interpolated data can cause issues,
the random forest model was used to predict the original 49
non-interpolated data points at each site as an additional check.

\subsection{Results}\label{results}

\subsubsection{Univariate summary}\label{univariate-summary}

Overall, the agricultural site had soil colour and geochemical
properties that exhibited lower variability and more symmetrical data
distributions as compared to the forested site
(Table~\ref{tbl-univariate-summary}). All 15 colour properties at both
sites displayed approximately symmetrical distributions. At the
agricultural site, all colour properties were characterized by low
coefficients of variation (CV), while the forested site showed slightly
greater variability, with 10 colour properties having low CVs and five
having moderate CVs.

Similarly, geochemical data at the agricultural site showed lower
variability and greater symmetry. Most elements were approximately
symmetrical, with only nine exhibiting moderate skewness and five highly
skewed (Table~\ref{tbl-univariate-summary}). Variability was also
limited, with the majority of elements having low CVs; 12 had moderate
CVs and five had high CVs. In contrast, the forested site showed greater
skewness and variability: seven elements exhibited moderate skewness, 14
were highly skewed, 28 had moderate CVs, six had high CVs, and two had
very high CVs.

\begin{longtable}[]{@{}ccccccc@{}}

\caption{\label{tbl-univariate-summary}Summary univariate statistics of
selected geochemical and colour soil properties for each site (n = 49).
Geochemical concentrations are reported in ppm, excecpt Ca and Fe(\%).}

\tabularnewline

\toprule\noalign{}
Property & Mean & SD & Max & Min & Skewness & CV \\
\midrule\noalign{}
\endhead
\bottomrule\noalign{}
\endlastfoot
\multicolumn{7}{@{}c@{}}{%
Agriculture} \\
Ca & 4.00 & 2.19 & 8.78 & 0.95 & 0.28 & 54.66 \\
Co & 8.76 & 0.83 & 10.60 & 7.50 & 0.52 & 9.48 \\
Cs & 0.75 & 0.15 & 1.07 & 0.47 & 0.18 & 19.93 \\
Fe & 1.92 & 0.09 & 2.11 & 1.71 & −0.25 & 4.70 \\
Li & 15.62 & 1.42 & 19.80 & 12.80 & 0.62 & 9.11 \\
La & 18.23 & 1.22 & 20.20 & 15.50 & −0.29 & 6.71 \\
Nb & 0.59 & 0.06 & 0.73 & 0.46 & 0.45 & 9.67 \\
Ni & 29.63 & 2.72 & 35.70 & 25.00 & 0.36 & 9.17 \\
Rb & 18.43 & 4.33 & 26.70 & 10.20 & 0.24 & 23.48 \\
Sr & 91.31 & 38.98 & 163.50 & 38.60 & 0.09 & 42.69 \\
a* & 3.38 & 0.32 & 4.15 & 2.59 & −0.03 & 9.53 \\
b* & 8.84 & 0.97 & 10.59 & 6.69 & −0.18 & 11.00 \\
c* & 9.47 & 1.02 & 11.32 & 7.17 & −0.19 & 10.74 \\
h* & 1.20 & 0.01 & 1.23 & 1.18 & 0.19 & 1.12 \\
x & 0.47 & 0.00 & 0.48 & 0.47 & 0.06 & 0.46 \\
\multicolumn{7}{@{}c@{}}{%
Forest} \\
Ca & 1.89 & 1.53 & 5.46 & 0.47 & 1.07 & 81.12 \\
Co & 6.76 & 1.39 & 9.60 & 4.00 & 0.03 & 20.62 \\
Cs & 0.55 & 0.12 & 0.78 & 0.34 & 0.25 & 21.73 \\
Fe & 1.18 & 0.13 & 1.46 & 0.83 & −0.58 & 11.24 \\
Li & 6.47 & 0.90 & 8.60 & 4.30 & −0.02 & 13.89 \\
La & 15.00 & 2.60 & 21.80 & 10.30 & 0.33 & 17.31 \\
Nb & 0.37 & 0.06 & 0.56 & 0.17 & −0.68 & 17.10 \\
Ni & 18.09 & 3.90 & 28.00 & 11.00 & 0.33 & 21.55 \\
Rb & 13.83 & 1.85 & 18.10 & 9.90 & 0.27 & 13.40 \\
Sr & 32.43 & 12.60 & 64.20 & 15.30 & 0.98 & 38.87 \\
a* & 5.73 & 0.41 & 6.56 & 4.41 & −0.38 & 7.10 \\
b* & 12.47 & 2.01 & 15.91 & 8.02 & 0.22 & 16.11 \\
c* & 13.74 & 1.94 & 17.00 & 9.15 & 0.15 & 14.15 \\
h* & 1.13 & 0.05 & 1.23 & 1.06 & 0.34 & 4.13 \\
x & 0.49 & 0.00 & 0.49 & 0.48 & −0.21 & 0.47 \\

\end{longtable}

\textsubscript{Source:
\href{https://alex-koiter.github.io/spatial-variability-soil-manuscript/notebooks/univariate_summary.qmd.html\#cell-tbl-univariate-summary}{Univariate
summary}}

The agricultural site has a relatively flat topography with an elevation
change of approximately 3m, with the field draining toward a ditch in
the northeast corner. The forested site has a relatively more complex
topography, with a channel flowing from the southwest toward the
northeast and an overall elevation difference of 18 m across the site.
The mean plan and profile curvature measurements for both sites are near
zero indicating an area of sediment transit and not accumulation or
erosion (Table~\ref{tbl-univariate2-summary}). The agricultural site had
a higher SAGA Wetness Index but the forested site had a larger range in
values and exhibited a higher degree of variability. The forested site
exhibited a smaller mean Relative Slope Position value (streams and
depressional areas) and a smaller Vertical Distance to Channel Network,
and for both terrain attributes a greater variability as compared to the
agricultural reflecting the presence of the stream crossing the forested
site.

\begin{longtable}[]{@{}ccccccc@{}}

\caption{\label{tbl-univariate2-summary}Summary statistics for the
interpolated values (10m resolution) for selected geochemical and colour
soil properties and terrain attributes for each site. Geochemical
concentrations are reported in ppm, excecpt Ca and Fe(\%).}

\tabularnewline

\toprule\noalign{}
Property & Mean & SD & Max & Min & Skewness & CV \\
\midrule\noalign{}
\endhead
\bottomrule\noalign{}
\endlastfoot
\multicolumn{7}{@{}c@{}}{%
Agriculture} \\
Ca & 4.12 & 2.10 & 8.76 & 0.918 & 0.0727 & 51.0 \\
Co & 8.75 & 0.664 & 10.6 & 7.52 & 0.431 & 7.59 \\
Cs & 0.729 & 0.123 & 1.07 & 0.458 & 0.376 & 16.9 \\
Fe & 1.92 & 0.0644 & 2.10 & 1.73 & −0.450 & 3.36 \\
Li & 15.7 & 1.16 & 19.3 & 13.2 & 0.551 & 7.38 \\
La & 18.2 & 0.817 & 19.8 & 16.5 & −0.268 & 4.49 \\
Nb & 0.593 & 0.0550 & 0.740 & 0.459 & 0.569 & 9.27 \\
Ni & 29.9 & 2.23 & 34.5 & 26.3 & −0.0100 & 7.46 \\
Rb & 18.0 & 3.94 & 26.1 & 11.5 & 0.498 & 21.8 \\
Sr & 93.4 & 38.6 & 167 & 36.3 & 0.00105 & 41.3 \\
a* & 3.34 & 0.211 & 3.83 & 2.88 & 0.0621 & 6.33 \\
b* & 8.73 & 0.707 & 10.2 & 6.98 & −0.162 & 8.10 \\
c* & 9.34 & 0.762 & 11.0 & 7.41 & −0.158 & 8.15 \\
h* & 1.20 & 0.00977 & 1.23 & 1.18 & −0.0603 & 0.811 \\
x & 0.473 & 0.00149 & 0.477 & 0.470 & −0.0168 & 0.314 \\
Plan Curvature & 1.65~×~10\textsuperscript{−6} &
1.36~×~10\textsuperscript{−4} & 6.57~×~10\textsuperscript{−4} &
−5.07~×~10\textsuperscript{−4} & 3.54~×~10\textsuperscript{−1} &
8.24~×~10\textsuperscript{3} \\
Profile Curvature & −7.64~×~10\textsuperscript{−6} &
1.53~×~10\textsuperscript{−4} & 5.83~×~10\textsuperscript{−4} &
−6.47~×~10\textsuperscript{−4} & 9.51~×~10\textsuperscript{−2} &
−2.00~×~10\textsuperscript{3} \\
SAGA Wetness Index & 9.64 & 0.704 & 11.2 & 7.77 & −0.122 & 7.30 \\
Catchment Area & 475 & 1,010 & 10,100 & 4.35 & 4.76 & 213 \\
Rel. Slope Position & 0.718 & 0.288 & 1.20 & 0.0221 & −0.946 & 40.1 \\
Vert. Dist. Channel & 5.98~×~10\textsuperscript{−2} &
4.10~×~10\textsuperscript{−2} & 2.92~×~10\textsuperscript{−1} &
4.25~×~10\textsuperscript{−3} & 1.21 & 6.85~×~10\textsuperscript{1} \\
Elevation & 310 & 0.593 & 312 & 309 & 0.615 & 0.191 \\
\multicolumn{7}{@{}c@{}}{%
Forest} \\
Ca & 1.88 & 0.769 & 3.61 & 0.787 & 0.202 & 40.8 \\
Co & 6.80 & 0.632 & 8.66 & 4.93 & −0.200 & 9.30 \\
Cs & 0.551 & 0.0737 & 0.714 & 0.423 & 0.297 & 13.4 \\
Li & 6.43 & 0.694 & 8.46 & 4.39 & −0.136 & 10.8 \\
La & 15.0 & 1.57 & 18.5 & 11.5 & −0.0324 & 10.4 \\
Nb & 0.370 & 0.0356 & 0.440 & 0.278 & −0.436 & 9.64 \\
Ni & 18.2 & 2.49 & 24.9 & 14.3 & 0.314 & 13.7 \\
Sr & 31.6 & 8.50 & 53.1 & 18.1 & 0.716 & 26.9 \\
h* & 1.13 & 0.0371 & 1.22 & 1.06 & 0.257 & 3.27 \\
Plan Curvature & 3.97~×~10\textsuperscript{−4} &
3.27~×~10\textsuperscript{−3} & 2.89~×~10\textsuperscript{−2} &
−2.62~×~10\textsuperscript{−2} & 7.91~×~10\textsuperscript{−1} &
8.22~×~10\textsuperscript{2} \\
Profile Curvature & −1.83~×~10\textsuperscript{−4} &
9.47~×~10\textsuperscript{−3} & 6.37~×~10\textsuperscript{−2} &
−7.37~×~10\textsuperscript{−2} & −5.31~×~10\textsuperscript{−1} &
−5.18~×~10\textsuperscript{3} \\
SAGA Wetness Index & 6.00 & 0.988 & 8.48 & 2.21 & −0.430 & 16.5 \\
Catchment Area & 571 & 1,940 & 25,400 & 3.44 & 6.60 & 339 \\
Rel. Slope Position & 0.222 & 0.232 & 0.993 & 0.00617 & 1.56 & 105 \\
Vert. Dist. Channel & 4.15~×~10\textsuperscript{−1} &
4.43~×~10\textsuperscript{−1} & 3.66 & 2.02~×~10\textsuperscript{−2} &
2.96 & 1.07~×~10\textsuperscript{2} \\
Elevation & 369 & 3.34 & 377 & 359 & −0.184 & 0.904 \\

\end{longtable}

\textsubscript{Source:
\href{https://alex-koiter.github.io/spatial-variability-soil-manuscript/notebooks/univariate_summary.qmd.html\#cell-tbl-univariate2-summary}{Univariate
summary}}

\subsubsection{Spatial analysis}\label{spatial-analysis}

Soil colour and geochemical composition varied across both sites. In the
agricultural field, all 15 soil color and geochemical properties
exhibited spatial autocorrelation with most properties demonstrating a
strong spatial dependency (Table~\ref{tbl-geocol-semivariogram}). Some
of the soil properties presented a pattern that roughly matches (e.g.,
Rb, Cs) or mirrors (e.g., Ca, Sr) the overall topography of the site
with a gradation between the highest point in the south-west corner
towards the lowest points in the north-east (Figure~\ref{fig-ag_map}).
Other properties appear to have more localized highs and low
concentrations/values (e.g., \emph{c*, h}). The geochemical
concentrations of Ca and Rb had the largest range values and as result
displayed a less patchy distribution across the site. The nugget (Co)
was small for all soil properties (\textless1.5), and Sr had an
exceptionally large sill value (900).

At the forested site, the geochemical concentrations of Fe and Rb, along
with the color properties \emph{a*, b*, c*,} and \emph{x} showed no
spatial autocorrelation and were excluded from further analysis and four
and five properties exhibiting strong and moderate spatial dependency,
respectively (Table~\ref{tbl-geocol-semivariogram}). In comparison to
the agricultural site, the soil properties at the forested site
displayed a more moderate spatial dependency. The nugget (Co) was
generally small for most soil properties (\textless2) with the exception
of La and Ni. The range values were similar across the different soil
properties and fell between 176 and 298 m. Overall, the influence of the
channel and floodplain environment can be seen in the pattern of the
nine soil properties (Figure~\ref{fig-forest_map}).

\begin{longtable}[]{@{}cccccccc@{}}

\caption{\label{tbl-geocol-semivariogram}Geostatistical parameters of
the fitted semivariogram models of selected colour and geochemical
properties within the agricultural and forested sites.}

\tabularnewline

\toprule\noalign{}
Property & Kriging Type{\textsuperscript{1}} & Nugget (Co) & Sill (Co +
C) & C/(C + Co) (\%) & Range (m) & r{\textsuperscript{2}} & Spatial
Class{\textsuperscript{2}} \\
\midrule\noalign{}
\endhead
\midrule\noalign{}
\multicolumn{8}{@{}c@{}}{%
{\textsuperscript{1}} Models are all isotropic.} \\
\multicolumn{8}{@{}c@{}}{%
{\textsuperscript{2}} Strong spatial dependency (C/(C + Co) \%
\textgreater75); Moderate spatial dependency (C/(C + Co) \% between 75
and 25); Low spatial dependency (C/(C + Co) \% \textless25).} \\
\bottomrule\noalign{}
\endlastfoot
\multicolumn{8}{@{}c@{}}{%
Agriculture} \\
Ca & Universal & 0.0 & 7.2 & 100 & 580 & 0.9 & Strong \\
Co & Simple & 0.0 & 0.7 & 100 & 208 & 0.4 & Strong \\
Cs & Ordinary & 0.0 & 0.0 & 100 & 210 & 0.5 & Strong \\
Fe & Ordinary & 0.0 & 0.0 & 100 & 185 & 0.2 & Strong \\
Li & Universal & 0.3 & 1.5 & 81 & 185 & 0.6 & Strong \\
La & Simple & 0.4 & 1.0 & 56 & 308 & 0.5 & Moderate \\
Nb & Universal & 0.0 & 0.0 & 91 & 210 & 0.7 & Strong \\
Ni & Ordinary & 1.4 & 8.9 & 84 & 352 & 0.6 & Strong \\
Rb & Ordinary & 1.4 & 27.6 & 95 & 551 & 0.9 & Strong \\
Sr & Ordinary & 0.9 & 900.2 & 100 & 220 & 1.0 & Strong \\
a* & Ordinary & 0.4 & 1.0 & 59 & 288 & 0.3 & Moderate \\
b* & Simple & 0.2 & 0.9 & 83 & 199 & 0.3 & Strong \\
c* & Simple & 0.1 & 0.9 & 87 & 199 & 0.3 & Strong \\
h* & Simple & 0.0 & 1.1 & 100 & 185 & 0.2 & Strong \\
x & Simple & 0.4 & 1.0 & 58 & 220 & 0.1 & Moderate \\
\multicolumn{8}{@{}c@{}}{%
Forest} \\
Ca & Ordinary & 1.6 & 2.7 & 41 & 269 & 0.2 & Moderate \\
Co & Ordinary & 0.0 & 2.1 & 100 & 298 & 0.1 & Strong \\
Cs & Ordinary & 0.0 & 0.0 & 83 & 237 & 0.2 & Strong \\
Li & Ordinary & 0.0 & 0.8 & 100 & 222 & 0.3 & Strong \\
La & Ordinary & 3.1 & 7.4 & 59 & 176 & 0.1 & Moderate \\
Nb & Ordinary & 0.0 & 0.0 & 51 & 224 & 0.2 & Moderate \\
Ni & Universal & 6.7 & 15.8 & 57 & 187 & 0.2 & Moderate \\
Sr & Simple & 0.4 & 1.0 & 65 & 229 & 0.4 & Moderate \\
h* & Universal & 0.0 & 0.0 & 100 & 230 & 0.3 & Strong \\

\end{longtable}

\textsubscript{Source:
\href{https://alex-koiter.github.io/spatial-variability-soil-manuscript/notebooks/semivariogram.qmd.html\#cell-tbl-geocol-semivariogram}{Semivariograms}}

\begin{figure}[H]

\centering{

\pandocbounded{\includegraphics[keepaspectratio]{index_files/figure-latex/notebooks-property_maps-fig-ag_map-output-1.png}}

}

\caption{\label{fig-ag_map}Kriged maps of select colour and geochemical
properties and elevation across the agricultural site.}

\end{figure}%

\textsubscript{Source:
\href{https://alex-koiter.github.io/spatial-variability-soil-manuscript/notebooks/property_maps.qmd.html\#cell-fig-ag_map}{Soil
property mapping}}

\begin{figure}[H]

\centering{

\pandocbounded{\includegraphics[keepaspectratio]{index_files/figure-latex/notebooks-property_maps-fig-forest_map-output-1.png}}

}

\caption{\label{fig-forest_map}Kriged map of select colour and
geochemical properties and elevation across the forested site.}

\end{figure}%

\textsubscript{Source:
\href{https://alex-koiter.github.io/spatial-variability-soil-manuscript/notebooks/property_maps.qmd.html\#cell-fig-forest_map}{Soil
property mapping}}

Across both sites, there was a significant (p \textless{} 0.05)
correlation between the selected soil properties and the terrain
attributes, with the exception of the plan and profile curvature
attributes (\textbf{?@supptab-correlation2}). The elevation attribute
generally had higher correlation coefficients; however, the direction
and strength of the correlation did vary between both site and soil
property. Overall, the random forest regression models exhibited
relatively strong predictive performance, with the models better
performing at the agricultural site compared to the forested site
(Table~\ref{tbl-rf-summary}). With the exception of the Ni concentration
and \emph{x} colour values at the agricultural site, elevation was
consistently the terrain attribute that provided the greatest predictive
power (Figure~\ref{fig-rf-results}). SAGA Wetness and relative slope
position were generally the second and third most informative terrain
attributes. Plan curvature was consistently ranked least important
predictive terrain attribute.

\begin{longtable}[]{@{}
  >{\centering\arraybackslash}p{(\linewidth - 16\tabcolsep) * \real{0.1111}}
  >{\centering\arraybackslash}p{(\linewidth - 16\tabcolsep) * \real{0.1111}}
  >{\centering\arraybackslash}p{(\linewidth - 16\tabcolsep) * \real{0.1111}}
  >{\centering\arraybackslash}p{(\linewidth - 16\tabcolsep) * \real{0.1111}}
  >{\centering\arraybackslash}p{(\linewidth - 16\tabcolsep) * \real{0.1111}}
  >{\centering\arraybackslash}p{(\linewidth - 16\tabcolsep) * \real{0.1111}}
  >{\centering\arraybackslash}p{(\linewidth - 16\tabcolsep) * \real{0.1111}}
  >{\centering\arraybackslash}p{(\linewidth - 16\tabcolsep) * \real{0.1111}}
  >{\centering\arraybackslash}p{(\linewidth - 16\tabcolsep) * \real{0.1111}}@{}}

\caption{\label{tbl-rf-summary}Model summary and performance statistics
for the random forest regression using the training, validation, test
and original (non-interpolated) data sets.}

\tabularnewline

\toprule\noalign{}
\multirow{2}{=}{\begin{minipage}[b]{\linewidth}\raggedright
Property
\end{minipage}} &
\multicolumn{2}{>{\centering\arraybackslash}p{(\linewidth - 16\tabcolsep) * \real{0.2222} + 2\tabcolsep}}{%
\begin{minipage}[b]{\linewidth}\centering
Training
\end{minipage}} &
\multicolumn{2}{>{\centering\arraybackslash}p{(\linewidth - 16\tabcolsep) * \real{0.2222} + 2\tabcolsep}}{%
\begin{minipage}[b]{\linewidth}\centering
Validation
\end{minipage}} &
\multicolumn{2}{>{\centering\arraybackslash}p{(\linewidth - 16\tabcolsep) * \real{0.2222} + 2\tabcolsep}}{%
\begin{minipage}[b]{\linewidth}\centering
Test
\end{minipage}} &
\multicolumn{2}{>{\centering\arraybackslash}p{(\linewidth - 16\tabcolsep) * \real{0.2222} + 2\tabcolsep}@{}}{%
\begin{minipage}[b]{\linewidth}\centering
Original
\end{minipage}} \\
& \begin{minipage}[b]{\linewidth}\raggedright
\% Variance{\textsuperscript{1}}
\end{minipage} & \begin{minipage}[b]{\linewidth}\raggedright
MSE{\textsuperscript{2}}
\end{minipage} & \begin{minipage}[b]{\linewidth}\raggedright
\% Variance{\textsuperscript{1}}
\end{minipage} & \begin{minipage}[b]{\linewidth}\raggedright
MSE{\textsuperscript{2}}
\end{minipage} & \begin{minipage}[b]{\linewidth}\raggedright
R{\textsuperscript{2}}
\end{minipage} & \begin{minipage}[b]{\linewidth}\raggedright
MSE{\textsuperscript{2}}
\end{minipage} & \begin{minipage}[b]{\linewidth}\raggedright
R{\textsuperscript{2}}
\end{minipage} & \begin{minipage}[b]{\linewidth}\raggedright
MSE{\textsuperscript{2}}
\end{minipage} \\
\midrule\noalign{}
\endhead
\midrule\noalign{}
\multicolumn{9}{@{}>{\centering\arraybackslash}p{(\linewidth - 16\tabcolsep) * \real{1.0000} + 16\tabcolsep}@{}}{%
{\textsuperscript{1}} Percent variance explained} \\
\multicolumn{9}{@{}>{\centering\arraybackslash}p{(\linewidth - 16\tabcolsep) * \real{1.0000} + 16\tabcolsep}@{}}{%
{\textsuperscript{2}} Mean square error} \\
\bottomrule\noalign{}
\endlastfoot
\multicolumn{9}{@{}>{\centering\arraybackslash}p{(\linewidth - 16\tabcolsep) * \real{1.0000} + 16\tabcolsep}@{}}{%
Agriculture} \\
Ca & 91.6 & 0.37 & 91.8 & 0.36 & 0.91 & 0.38 & 0.95 & 0.23 \\
Co & 79.8 & 0.09 & 82.5 & 0.08 & 0.80 & 0.08 & 0.88 & 0.08 \\
Cs & 85.7 & 0.00 & 86.4 & 0.00 & 0.85 & 0.00 & 0.92 & 0.00 \\
Fe & 69.6 & 0.00 & 70.9 & 0.00 & 0.69 & 0.00 & 0.83 & 0.00 \\
Li & 59.3 & 0.54 & 59.8 & 0.53 & 0.64 & 0.51 & 0.88 & 0.24 \\
La & 93.0 & 0.05 & 93.1 & 0.04 & 0.93 & 0.05 & 0.96 & 0.03 \\
Nb & 57.3 & 0.00 & 59.1 & 0.00 & 0.55 & 0.00 & 0.71 & 0.00 \\
Ni & 93.1 & 0.34 & 93.7 & 0.33 & 0.93 & 0.34 & 0.95 & 0.25 \\
Rb & 95.3 & 0.73 & 96.1 & 0.64 & 0.95 & 0.79 & 0.98 & 0.39 \\
Sr & 93.5 & 97.22 & 93.6 & 93.97 & 0.93 & 105.59 & 0.97 & 44.77 \\
a* & 85.0 & 0.01 & 86.9 & 0.01 & 0.85 & 0.01 & 0.91 & 0.00 \\
b* & 72.5 & 0.14 & 75.3 & 0.12 & 0.72 & 0.15 & 0.89 & 0.09 \\
c* & 73.2 & 0.15 & 75.9 & 0.14 & 0.73 & 0.17 & 0.89 & 0.10 \\
h* & 58.3 & 0.00 & 58.6 & 0.00 & 0.56 & 0.00 & 0.73 & 0.00 \\
x & 73.3 & 0.00 & 73.6 & 0.00 & 0.69 & 0.00 & 0.82 & 0.00 \\
\multicolumn{9}{@{}>{\centering\arraybackslash}p{(\linewidth - 16\tabcolsep) * \real{1.0000} + 16\tabcolsep}@{}}{%
Forest} \\
Co & 39.1 & 0.24 & 42.9 & 0.23 & 0.48 & 0.21 & 0.77 & 0.29 \\
Cs & 64.1 & 0.00 & 67.1 & 0.00 & 0.66 & 0.00 & 0.86 & 0.00 \\
Li & 41.3 & 0.28 & 42.0 & 0.28 & 0.46 & 0.28 & 0.66 & 0.26 \\
La & 43.3 & 1.40 & 47.5 & 1.32 & 0.48 & 1.23 & 0.78 & 0.60 \\
Nb & 55.0 & 0.00 & 55.9 & 0.00 & 0.58 & 0.00 & 0.84 & 0.00 \\
Sr & 59.4 & 29.43 & 59.1 & 29.66 & 0.59 & 29.25 & 0.82 & 13.88 \\
h* & 58.8 & 0.00 & 60.3 & 0.00 & 0.62 & 0.00 & 0.86 & 0.00 \\

\end{longtable}

\textsubscript{Source:
\href{https://alex-koiter.github.io/spatial-variability-soil-manuscript/notebooks/RF_summary.qmd.html\#cell-tbl-rf-summary}{Random
Forest summary}}

\begin{figure}[H]

\centering{

\pandocbounded{\includegraphics[keepaspectratio]{index_files/figure-latex/notebooks-RF_summary-fig-rf-results-output-1.png}}

}

\caption{\label{fig-rf-results}Heat map of the Random Forest regression
results showing the ranking of the importance of terrain attributes
(based on \% increase in Mean Squared Error) in explaining the spatial
variabilty of selected colour and geochemical properties within the
agricultural and forested sites. Top panel shows an average ranking for
each site and across both sites.}

\end{figure}%

\textsubscript{Source:
\href{https://alex-koiter.github.io/spatial-variability-soil-manuscript/notebooks/RF_summary.qmd.html\#cell-fig-rf-results}{Random
Forest summary}}

\subsection{Discussion}\label{discussion}

\subsubsection{Variability of soil
properties}\label{variability-of-soil-properties}

Variability in soil geochemical properties have been studied at a range
of scales including continental \citep{drew2010}, regional
\citep{rattenbury2018}, watershed \citep{nanos2012}, hillslope/catena,
and farm field \citep{sun2021a}. The objectives of these studies
included addressing issues of pollution/contamination, providing
benchmark/baseline information, investigating pedological and weathering
properties and processes, and soil surveying and mapping
\citep{wilson2008}. Similarly, variability in soil colour, typically
using the Munsell colour system, is a commonly reported diagnostic
feature used in soil classification and ranges in spatial scale from
reconnaissance to detailed soil surveys and maps. For sediment
fingerprinting studies, these types of studies are often too
site-specific or focus on a smaller subset of soil properties to
effectively guide sample design to ensure the desired confidence is met
characterizing sources of sediment.

Data distributions in soil science commonly exhibit a positively skewed
distribution. This is likely due to several factors including that data
of this nature are a semi-bounded distribution, with a lower bound of
zero and no upper bound. Hot spots of soil processes, local variations
in soil forming factors, and soil/land management practices can also
lead to more extreme values \citep[e.g.,][]{vidon2010}. In many cases
the cumulative effects of these processes, factors, and practices are
multiplicative (i.e., interact) and not linearly additive, resulting in
a skewed data distribution. Lastly, the distribution of data will also
be a product of the scale of observation, number of samples, and
sampling design.

Soil colour properties exhibited a near-normal distribution with a low
CV which is consistent with claims that soil hue and value (Munsell
colour system) have a low CV \citep{pennock2008}. These data
distribution properties are ideal for statistical and environmental
modeling as it typically meets the model assumptions with out requiring
transformations. For example, in sediment source fingerprinting, soil
properties (i.e., fingerprints) are considered more reliable and robust
for use in unmixing models when they show large differences between
sources and low variability within each source. Additionally, most
mixing models assume fingerprint data are normally distributed.
\citep{lunamiño2024} demonstrated that soil colour coefficients a*, b*,
c*, h*, and x provided good discrimination between the agricultural and
forested sites, and the low CV and skewness values reported in
Table~\ref{tbl-univariate-summary} makes these colour properties ideal
fingerprints for sediment source apportionment.

The geochemical properties were more variable and skewed as compared to
the soil colour properties. For many trace elements, concentrations are
strongly correlated with the proportion of fine-grained material
(\textless2\,µm), due to its high specific surface area and enhanced
chemical reactivity \citep{horowitz1991}. However, in this study the
sand-size (\textgreater63\,µm) material was removed prior to analysis to
reduce the effects of grain-size on concentration. This likely resulted
in lower variability and less extreme concentrations as compared to
other studies that focus on bulk soil samples (\textless2 mm). In
particular, the forested site exhibited a greater amount of variability
which is likely due to the more complex topography and geomorphic
setting. The floodplain within the forested site is likely accumulating
shale-rich material derived from the Manitoba Escarpment which is
enriched in trace metals \citep{nicolas2011}. This creates a zone of
high concentrations relative to upland areas
Figure~\ref{fig-forest_map}. The forested site also had a higher and
much more variable soil organic matter content (\(\bar{x}\) = 8.5 \%, CV
= 51.9 \%) as compared to the agricultural site (\(\bar{x}\) = 11.6 \%,
CV = 16.1 \%), which similarly to the grain size distribution, can
influence the concentration of many major and trace elements
\citep{horowitz1991}. These results provide evidence that both land use
and landscape complexity both play a role in driving soil property
variability.

\subsubsection{Spatial distribution}\label{spatial-distribution}

The difference in the number of soil properties and the magnitude of the
spatial auto correlation between the two sites can be used in designing
an effect sampling campaign. The agricultural site, which has a simpler
topography and a higher degree of spatial autocorrelation, the range
values can be used to guide the distance between sampling points and a
grid-style sampling regime may be an effective approach. In contrast,
the forested site, which has a more complex geomorphic setting and a
lower degree of spatial autocorrelation, a stratified sampling design
may be the better approach. For example, at the forested site the
stratas could include near-stream and hillslope environments. In
situations where the soil properties of interest are not known or
selected \emph{a priori} (e.g., sediment fingerprinting) the differences
in their spatial autocorrelation are difficult to accommodate in the
sampling design. A sampling grid with irregular spacing, including
spacing less than 100m, would have provided information on the spatial
autocorrelation over shorter distances and reduced the uncertainty in
the interpolation of soil properties \citep{lark2018}.

Mapping the soil properties that have a moderate to high spatial
dependence can provide information on underlying soil forming processes
and properties. At both sites, to some extent, the patterns appear to
reflect the topography of the sites suggesting that geomorphic and
hydrologic processes and properties are likely driving the observed
patterns. Identifying patterns and understanding the underlying process
and properties that drive these patterns are important considerations
when designing as soil sampling campaign to successfully meet study
objectives, including characterizing soil properties of a field site. In
a related context, Koiter \citep{koiter2013a} discussed the issues
surrounding the use of a statistical only approach to selecting
fingerprints and that consideration of how fingerprints have developed
improves the robustness of the sediment fingerprinting approach.
However, local information on the spatial distribution of geochemical
and colour properties at field scales (\textless{} 1 km²) is often
unavailable, and the processes driving these patterns are also not well
documented or studied. When such information does exist, it typically
focuses on agronomically important properties \citep[e.g.,][]{mzuku2005}
or is used for soil classification
\citep[e.g.,][]{soilclassificationworkinggroup1998}. These datasets
usually include geochemical properties such as nitrogen (N), phosphorus
(P), potassium (K), sulfur (S), calcium (Ca), magnesium (Mg), sodium
(Na), iron (Fe), aluminum (Al), nitrate (NO₃⁻), carbonate (CO₃²⁻),
bicarbonate (HCO₃⁻), chloride (Cl⁻), and sulfate (SO₄²⁻). They may also
include colour characteristics, such as Munsell hue, value, and chroma,
as well as other soil properties like texture, organic matter content,
and pH. The lack of information on the wide range of soil properties
means the researchers are relying on other data, most often elevation,
for informing sampling designs.

\subsubsection{Terrain attributes and soil
properties}\label{terrain-attributes-and-soil-properties}

Both the correlation analysis and random forest regression identified
elevation as the most influential terrain attribute, followed by the
SAGA Wetness Index and relative slope position, in explaining most of
the observed variation in soil geochemical and colour properties. This
is consistent with the findings of Mashalaba \citep{mashalaba2020} who
also found that similar terrain attributes were important in predicting
a range of other soil properties including texture, bulk density, and
hydaulic conductivity. These attributes likely emerged as the most
important factor in explaining the observed variability as they are
strongly linked to a range of geomorphic and hydrologic process and
conditions {[}\citep{mello2022}; \citep{libohova2024}{]}. For example,
in eroded landscapes in the Prairie region of Canada, Ca concentrations
have been found to be higher in upper slope positions from erosion and
subsquent exposure of high-carbonate subsoil \citep{papiernik2005}. In
contrast, higher Ca concentrations have been noted in lower slope and
depressional areas due to higher solubility of many Ca-minerals (e.g.,
CaCO\textsubscript{3}) and the subsequent downslope transport in
solution and reduced leaching losses in these accumulation zones.
Landscape position can also have a strong influence on pedogenic
process; for example, the translocation of Fe and clay down the soil
profile is a diagnostic criteria used in classifying soils
\citep{stonehouse1971}. Soil colour also tends to change in a
predictable manner in relation to local relief. Tillage and water
erosion results in the net loss of darker organic-rich topsoil from
upper slope positions resulting in the exposure of the lighter subsoil
\citep{papiernik2005}. Moisture availability is also greater in the
lower slope and depressional areas resulting in increased organic matter
production resulting in darker organic-rich topsoil as compared to the
upper slope positions. There is also evidence that suggests that soil
texture varies with elevation and slope position, with coarser material
on upper slopes and finer material accumulating in lower positions
\citep{kokulan2018, cox2003}. Given the strong correlation of organic
matter and texture with soil geochemistry \citep{horowitz1991} and
colour \citep{viscarrarossel2009}, these properties may also help
explain the observed spatial patterns .

The relative importance of terrain attributes in explaining soil
property variability differs both among soil properties and between
sites. The land use and the overall geomorphic complexity differences
between the two study sites are likely interacting with terrain
attributes and influencing the patterns of soil properties and modifying
the nature of terrain attribute and soil property relationship. This
suggests that these relationships observed in this study may not be
broadly generalized. Similarly, information on how terrain attributes
influence the spatial distribution of many trace elements and soil
colour, beyond the Munsell system, at the field scale is limited in the
scientific literature. Additional variables including climate and
large-scale landscape features will also influence the observed patterns
of soil properties. As a result, using terrain attributes to guide soil
sampling or interpret spatial patterns of many soil properties remains
challenging.

The impact of sampling design at the field scale on the characterization
of soil properties can be substantial \citep{lunamiño2024}, which in
turn can affect the interpretation of data, modeling results, and the
conclusion drawn. High-quality LiDAR data or digital elevation models
(DEMs) are increasingly openly available in many regions and can be used
to create detailed terrain attribute maps. By incorporating terrain
attributes into the sampling framework, researchers can ensure that key
geomorphic and hydrologic gradients are adequately represented.
Ultimately, integrating terrain analysis into sediment source
fingerprinting is promising not only as a mechanism to improve the
quality of source characterization but to also better link source
material to downstream sediment.

\subsection{Conclusions}\label{conclusions}

Understanding the spatial variability and distribution of soil
geochemical and colour properties at a field-scale is important for
agricultural and environmental research, monitoring, modeling, and
management practices. This study conducted both univariate and spatial
analyses of a suite of soil geochemical and colour properties at two
sites with contrasting land uses. The agricultural site, characterized
by gently sloping topography, exhibited lower coefficients of variation,
approximately normal data distributions, and moderate to strong spatial
autocorrelation across most measured properties. In contrast, the
forested site featured more geomorphologically complex terrain, with
greater variability in soil properties, data distributions that more
frequently deviated from normality, and fewer properties exhibiting
spatial autocorrelation. Despite these differences, random forest
regression consistently identified elevation, the SAGA Wetness Index,
and relative slope position as the three most important terrain
attributes explaining the observed variability.

These findings underscore the role of topographic controls on many soil
property distributions, regardless of land use. However, the strength
and direction of the relationship between terrain attributes and soil
property results were inconsistent between both site and soil property.
While the study was limited to two sites, the approach demonstrates the
value of integrating tools like random forest regression with spatial
data to better understand soil-landscape relationships. Future research
should expand to broader landscapes and incorporate additional
biophysical variables to improve generalizability. Overall, this work
highlights how terrain-driven spatial patterns can inform more targeted
soil sampling, modeling, and land management strategies.

\subsection*{Acknowledgments}\label{acknowledgments}
\addcontentsline{toc}{subsection}{Acknowledgments}

Special thanks and recognition for the field and technical support from
A. Avila and the Riding Mountain National Park personnel.

\subsection*{Statements and
declarations}\label{statements-and-declarations}
\addcontentsline{toc}{subsection}{Statements and declarations}

\subsubsection*{Funding}\label{funding}
\addcontentsline{toc}{subsubsection}{Funding}

This research was supported by the Natural Sciences and Engineering
Research Council of Canada Discovery Grant - From source to sink:
Investigating the linkages between sources of sediment and downstream
water quality in Canadian watersheds - awarded to AJK
(RGPIN-2019-05273).

\subsubsection*{Competing interests}\label{competing-interests}
\addcontentsline{toc}{subsubsection}{Competing interests}

The authors have no competing interests to declare that are relevant to
the content of this article.

\subsubsection*{Data and code
availability}\label{data-and-code-availability}
\addcontentsline{toc}{subsubsection}{Data and code availability}

Data and source code for analysis and manuscript available on GitHub:
\url{https://github.com/alex-koiter/sampling-design-manuscript}

\subsubsection*{Author contributions}\label{author-contributions}
\addcontentsline{toc}{subsubsection}{Author contributions}

\textbf{M Luna Miño} Methodology; Investigation; Data curation; Formal
analysis; Writing - Original Draft; Writing - Review \& Editing

\textbf{A Koiter}: Conceptualization; Funding acquisition; Methodology;
Investigation; Data curation; Formal analysis; Visualization; Writing -
Original Draft; Writing -- review and editing; Software; Project
administration

\textbf{T Lychuk}: Methodology; Formal analysis; Writing - Review \&
Editing

\textbf{A Waddel} Methodology; Formal analysis; Writing - Review \&
Editing

\textbf{A Moulin} Methodology; Formal analysis; Writing - Review \&
Editing

\subsection*{References}\label{references}
\addcontentsline{toc}{subsection}{References}

\renewcommand{\bibsection}{}
\bibliography{references.bib}

\subsection*{Supplemental figures}\label{supplemental-figures}
\addcontentsline{toc}{subsection}{Supplemental figures}

\begin{suppfig}

\centering{

\includegraphics[width=1\linewidth,height=\textheight,keepaspectratio]{images/colour_summary.png}

}

\caption{\label{suppfig-colour_summary}Summary statistics of all
measured colour soil properties at both sites. Error bars represent 1SD
and the numeric values indicate the CV.}

\end{suppfig}%

\begin{suppfig}

\centering{

\includegraphics[width=1\linewidth,height=\textheight,keepaspectratio]{images/geo_summary.png}

}

\caption{\label{suppfig-geo_summary}Summary statistics of all measured
geochemical soil properties at both sites. Error bars represent 1SD and
the numeric values indicate the CV.}

\end{suppfig}%

\subsection*{Supplemental tables}\label{supplemental-tables}
\addcontentsline{toc}{subsection}{Supplemental tables}

\begin{supptab}

\caption{\label{supptab-abbrev}Description of spectral reflectance
colour coefficients used as fingerprints. Reproduced from Boudreault et
al.~(2018)}

\centering{

\begin{longtable*}[]{@{}
  >{\raggedright\arraybackslash}p{(\linewidth - 4\tabcolsep) * \real{0.2660}}
  >{\raggedright\arraybackslash}p{(\linewidth - 4\tabcolsep) * \real{0.5532}}
  >{\raggedright\arraybackslash}p{(\linewidth - 4\tabcolsep) * \real{0.1702}}@{}}
\toprule\noalign{}
\begin{minipage}[b]{\linewidth}\raggedright
\textbf{Colour space model}
\end{minipage} & \begin{minipage}[b]{\linewidth}\raggedright
Parameter
\end{minipage} & \begin{minipage}[b]{\linewidth}\raggedright
Abbreviation
\end{minipage} \\
\midrule\noalign{}
\endhead
\bottomrule\noalign{}
\endlastfoot
RGB & Red & R \\
RGB & Green & G \\
RGB & Blue & B \\
CIE xyY & Chromatic Coordinate x & x \\
CIE xyY & Chromatic Coordinate y & y \\
CIE xyY & Brightness & Y \\
CIE XYZ & Virtual component X & X \\
CIE XYZ & Virtual component Z & Z \\
CIE LAB & Metric lightness function & L \\
CIE LAB & Chromatic coordinate opponent red--green scales & \emph{a*} \\
CIE LAB & Chromatic coordinate opponent red--green scales & \emph{b*} \\
CIE LUV & Chromatic coordinate opponent blue--yellow scales &
\emph{u*} \\
CIE LUV & Chromatic oordinate opponent red--green scales & \emph{v*} \\
CIE LCH & CIE hue & \emph{c*} \\
CIE LCH & CIE chroma & \emph{h*} \\
\end{longtable*}

}

\end{supptab}%

\begin{supptab}

\caption{\label{supptab-terrain}Terrain attribute descriptions}

\centering{

\begin{longtable*}[]{@{}
  >{\raggedright\arraybackslash}p{(\linewidth - 2\tabcolsep) * \real{0.0805}}
  >{\raggedright\arraybackslash}p{(\linewidth - 2\tabcolsep) * \real{0.9195}}@{}}
\toprule\noalign{}
\begin{minipage}[b]{\linewidth}\raggedright
Terrain Attribute
\end{minipage} & \begin{minipage}[b]{\linewidth}\raggedright
Description
\end{minipage} \\
\midrule\noalign{}
\endhead
\bottomrule\noalign{}
\endlastfoot
Elevation & Meters above sea level \\
Plan Curvature & Across slope curvature \\
Profile Curvature & Down slope curvature \\
SAGA Wetness Index & Similar to the `Topographic Wetness Index' (TWI),
but it is based on a modified catchment area calculation, which does not
think of the flow as very thin film. As result it predicts for cells
situated in valley floors with a small vertical distance to a channel a
more realistic, higher potential soil moisture compared to the standard
TWI calculation \\
Catchment Area & Area of upslope contributing area \\
Relative Slope Position & A value between 0 and 1 illustrating the
position of a pixel within the landscape with values approaching 0
indicating streams to pits, and values approaching 1 indicating upper
slope positions to peaks \\
Vertical Distance to Channel &
\begin{minipage}[t]{\linewidth}\raggedright
The vertical distance to a channel network base level. The algorithm
consists of two major steps:

1. Interpolation of a channel network base level elevation\\
2. Subtraction of this base level from the original elevations\strut
\end{minipage} \\
\end{longtable*}

}

\end{supptab}%





\end{document}
